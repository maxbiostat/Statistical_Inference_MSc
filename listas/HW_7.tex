\documentclass[a4paper,10pt, notitlepage]{report}
\usepackage[utf8]{inputenc}
\usepackage{natbib}
\usepackage{amssymb}
\usepackage{amsmath}
\usepackage{enumitem}
\usepackage{xcolor}
\usepackage{cancel}
\usepackage{mathtools}
\usepackage[portuguese]{babel}

%%%%%%%%%%%%%%%%%%%% Notation stuff
\newcommand{\pr}{\operatorname{Pr}} %% probability
\newcommand{\vr}{\operatorname{Var}} %% variance
\newcommand{\rs}{X_1, X_2, \ldots, X_n} %%  random sample
\newcommand{\ods}{X_{(1)}, X_{(2)}, \ldots, X_{(n)} } %%  ordered sample
\newcommand{\irs}{X_1, X_2, \ldots} %% infinite random sample
\newcommand{\rsd}{x_1, x_2, \ldots, x_n} %%  random sample, realised
\newcommand{\bX}{\boldsymbol{X}} %%  random sample, contracted form (bold)
\newcommand{\bx}{\boldsymbol{x}} %%  random sample, realised, contracted form (bold)
\newcommand{\bT}{\boldsymbol{T}} %%  Statistic, vector form (bold)
\newcommand{\bt}{\boldsymbol{t}} %%  Statistic, realised, vector form (bold)
\newcommand{\emv}{\hat{\theta}}
\DeclarePairedDelimiter\ceil{\lceil}{\rceil}
\DeclarePairedDelimiter\floor{\lfloor}{\rfloor}
\newcommand{\rpl}{\mathbb{R}_+}

% Title Page
\title{Exercícios: Testes de hipóteses}
\author{Disciplina: Inferência Estatística (MSc) \\ Instrutor: Luiz Carvalho \\ Monitor: Isaque Pim}
\date{Dezembro/2023}

\begin{document}
\maketitle


\paragraph{Notação:} Como convenção adotamos $\mathbb{R} = (-\infty, \infty)$, $\rpl = (0, \infty)$ e $\mathbb{N} = \{1, 2, \ldots \}$.

\paragraph{Motivação:} Testes de hipóteses são ferramentas matemáticas para avaliar a compatibilidade de dados observados com afirmações sobre a configuração do espaço de medidas que os possam ter gerado. 
Estes instrumentos encontram vasta aplicação nas ciências aplicadas, onde afirmações científicas sobre o funcionamento do mundo são traduzidas em afirmações matemáticas do tipo discutido acima. 
Nesta lista de exercícios vamos ver a construção de testes uniformemente mais poderosos, de razão de verossimilhanças e o cálculo do valor $p$.

\paragraph{Dos livros-texto:}

\begin{itemize}
    \item[a)] KN, Ch12: 10, 17, 25, 32;  
    \item[b)] CB, Ch8: 8.1, 8.2, 8.3, 8.5, 8.8, 8.18, 8.20, 8.28, 8.31, 8.38;
    \item[b)] CB, Ch9: 9.5, 9.6, 9.44.
\end{itemize}

\paragraph{Extra:}

\begin{enumerate} 
\item (\textbf{Existência de um TUMP}) Suponha que $\rs$ são amostra aleatória de uma distribuição normal com média $\mu$ desconhecida e variância $\sigma$ conhecida.
Considere testar $H_0: \mu = \mu_0$ versus $H_1 = \mu \neq \mu_0$. 
Mostre que não existe teste uniformemente mais poderoso neste caso.
\textbf{Dica}: Construa um teste da forma
\begin{equation*}
    \psi_0(\bX) = \begin{cases}
    0, \bar{X}_n \geq - \frac{\sigma z_\alpha}{\sqrt{n}} + \mu_0,\\
    1, c.c.,
    \end{cases}
\end{equation*}
e investigue o que acontece num ponto $\theta_1 < \mu_0$. 
Depois, construa um segundo teste, complementar e use o Teorema de Neyman-Pearson para concluir que nenhum TUMP pode existir.
\item (\textbf{Testes aleatorizados}) Tome $X_1, \ldots, X_{10}$ amostra aleatória de uma distribuição Bernoulli com parâmetro $\theta \in (0, 1)$, desconhecido.
Considere testar $H_0: \theta \leq 1/2$ \textit{versus} $H_1: \theta > 1/2$.
\begin{enumerate}
 %KN pg 136
    \item  Escreva um teste de razão de verossimilhanças para este caso;
    \item  Mostre que não é possível atingir qualquer nível de significância $\alpha_0 \in (0, 1)$
    \item Proponha um \textit{teste aleatorizado} que atinja um tamanho $\alpha_0 = 0.05$.
    Isto é, proponha um teste aleatório $\psi_A$ tal que a decisão de aceitar ou não $H_0$ dependa do lançamento de uma moeda com probabilidade de sucesso $\omega_A$, e dê uma forma para $\omega_A$.
\end{enumerate}
   % \item \textbf{Desafio}:      
 % \textbf{Dica:} 
\end{enumerate}
% \newpage
% \bibliographystyle{apalike}
% \bibliography{refs}

\end{document}          

