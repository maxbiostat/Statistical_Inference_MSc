\documentclass[a4paper,10pt, notitlepage]{report}
\usepackage[utf8]{inputenc}
\usepackage{natbib}
\usepackage{amssymb}
\usepackage{amsmath}
\usepackage{enumitem}
\usepackage{xcolor}
\usepackage{cancel}
\usepackage{mathtools}
\usepackage[portuguese]{babel}

%%%%%%%%%%%%%%%%%%%% Notation stuff
\newcommand{\pr}{\operatorname{Pr}} %% probability
\newcommand{\vr}{\operatorname{Var}} %% variance
\newcommand{\rs}{X_1, X_2, \ldots, X_n} %%  random sample
\newcommand{\ods}{X_{(1)}, X_{(2)}, \ldots, X_{(n)} } %%  ordered sample
\newcommand{\irs}{X_1, X_2, \ldots} %% infinite random sample
\newcommand{\rsd}{x_1, x_2, \ldots, x_n} %%  random sample, realised
\newcommand{\bX}{\boldsymbol{X}} %%  random sample, contracted form (bold)
\newcommand{\bx}{\boldsymbol{x}} %%  random sample, realised, contracted form (bold)
\newcommand{\bT}{\boldsymbol{T}} %%  Statistic, vector form (bold)
\newcommand{\bt}{\boldsymbol{t}} %%  Statistic, realised, vector form (bold)
\newcommand{\emv}{\hat{\theta}}
\DeclarePairedDelimiter\ceil{\lceil}{\rceil}
\DeclarePairedDelimiter\floor{\lfloor}{\rfloor}
\newcommand{\rpl}{\mathbb{R}_+}

% Title Page
\title{Exercícios: Inferência aproximada}
\author{Disciplina: Inferência Estatística (MSc) \\ Instrutor: Luiz Carvalho \\ Monitor: Isaque Pim}
\date{Dezembro/2023}

\begin{document}
\maketitle


\paragraph{Notação:} Como convenção adotamos $\mathbb{R} = (-\infty, \infty)$, $\rpl = (0, \infty)$ e $\mathbb{N} = \{1, 2, \ldots \}$.

\paragraph{Motivação:} Em muitas aplicações de inferência paramétrica estamos interessados em estimar quantidades de interesse $g(\theta)$ em vez do parâmetro $\theta$ diretamente.
Quando vale um teorema central do limite para uma classe de estimadores $(\hat{\theta}_n)_{n\geq1}$, é possível computar a distribuição assintótica de $g(\hat{\theta}_n)$, sujeito a algumas restrições em $g$.
Ao conjunto de técnicas envolvidas neste procedimento chamamos \textit{método Delta}.
Nesta lista veremos algumas aplicações do método Delta, em particular para encontrar as chamadas \textit{transformações estabilizadoras da variância}.

\section*{Transformações estabilizadoras da variância}

Suponha que $E_\theta[X] = \mu(\theta)$ é a nossa quantidade de interesse.
 Aplicando o teorema central do limite, temos
 \begin{equation}
   \sqrt{n}\left(\bar{X}_n -\mu\right)  \xrightarrow{d} \operatorname{Normal}(0, \sigma^2(\theta)).
 \end{equation}
 O problema é que $\sigma^2(\theta) = v(\mu)$ é função de $\mu$, a quantidade que queremos estimar. 
 Idealmente, gostaríamos que a distribuição-limite fosse independente de $\mu$.
 Para tanto, queremos uma transformação $g : \mathcal{X} \to \mathbb{R}$ tal que
  \begin{equation}
   \sqrt{n}\left(g(\bar{X}_n) -g(\mu)\right)  \xrightarrow{d} \operatorname{Normal}\left(0, [g^\prime(\mu)]^2v(\mu)\right),
 \end{equation}
 tenha variância constante com respeito a $\mu$.
 Ou seja, queremos $g$ tal que $g^\prime(\mu)v(\mu) = a$ para todo $\mu$.
 Se $g$ cumpre estes requisitos, dizemos que é uma \textbf{transformação estabilizadora da variância}.

\paragraph{Dos livros-texto:}

\begin{itemize}
    \item[a)] KN, Ch8: 4a;  
    \item[b)] CB, Ch5: 5.66. 
\end{itemize}

\paragraph{Extra:}

\begin{enumerate} 
\item (\textbf{Família exponencial}) Suponha que $\rs$ é uma amostra aleatória de uma família dominada $P_\theta$ da família exponencial com um parâmetro, sob parametrização canônica, isto é, a densidade é 
\begin{equation*}
    p_\eta(x) = h(x)\exp\left[\eta(\theta) T(x) - B(\theta)\right].
\end{equation*}
Seja $\hat{\eta}(\bX_n)$ o estimador de máxima verossimilhança  (EMV) de $\eta(\theta)$ sob este modelo.
\begin{enumerate}
 %KN pg 136
    \item Encontre a distribuição assintótica de 
\begin{equation*}
    \sqrt{n}\{\hat{\eta}(\bX_n)-\eta\},
\end{equation*}
e discuta se o EMV atinge a cota inferior de Cramér-Rao \textit{assintoticamente}.
%KN, pg 406
    \item  Considere estimar uma quantidade de interesse $g(\theta)$. 
    Assumindo que $g$ é contínua, diferenciável e com derivada diferente de zero, encontre um intervalo de confiança aproximado (assintótico) para $\lambda = g(\theta)$ com nível $1-\alpha$, $\alpha \in (0, 1)$.
\end{enumerate}

 \item (\textbf{Binomial}) Seja $\rs$ uma amostra aleatória de uma distribuição binomial com número de tentativas $K \in \mathbb{N}$ conhecido e probabilidade de sucesso $\theta \in (0, 1)$ desconhecida.
 \begin{enumerate}
     \item Considere estimar $w_1(\theta) = \theta(1-\theta)$.
     Seja $\delta_1(\bX_n)$ o EMV para $w_1$.
     Encontre a distribuição assintótica de 
     \begin{equation*}
        \sqrt{n}\left\{\delta_1(\bX_n) - w_1(\theta)\right\}.
     \end{equation*}
     \item O que acontece com a distribuição assintótica de $\delta_1(\bX_n)$ quando $\theta = 1/2$? Encontre sequências $(a_n)_{n\geq1}$ e $(b_n)_{n\geq1}$ tal que 
     \begin{equation*}
         a_n\left(\delta_1(\bX_n) - b_n\right) \xrightarrow{d} X,
     \end{equation*}
     com $X$ distribuído Bernoulli com parâmetro $\theta = 1/2$.
     \textbf{Dica:} faça $Y_n = X_n-(n/2)$ e aplique o teorema central do limite a $Y_n$. 
     \item Mostre que $g(x) = \arcsin(\sqrt{x})$ é uma transformação estabilizadora da variância.
 \end{enumerate}
 \item (\textbf{Normal com variância desconhecida}) 
 Seja $\rs$ uma amostra aleatória de uma distribuição normal com média 0 e variância $\theta >0$ desconhecida. 
 \begin{enumerate}
         \item Encontre $\vr_\theta(\bar{X}_n)$;
         \item Mostre que uma transformação estabilizadora da variância neste caso é $g(x) = \log(x) + c$, $c \in \mathbb{R}$.
 \end{enumerate}
 
  % \item \textbf{Desafio}:      
 % \textbf{Dica:} 
\end{enumerate}
% \newpage
% \bibliographystyle{apalike}
% \bibliography{refs}

\end{document}          

