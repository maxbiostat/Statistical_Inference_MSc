\documentclass[a4paper,10pt, notitlepage]{report}
\usepackage[utf8]{inputenc}
\usepackage{natbib}
\usepackage{amssymb}
\usepackage{amsmath}
\usepackage{enumitem}
\usepackage{xcolor}
\usepackage{cancel}
\usepackage{mathtools}
\usepackage[portuguese]{babel}

%%%%%%%%%%%%%%%%%%%% Notation stuff
\newcommand{\pr}{\operatorname{Pr}} %% probability
\newcommand{\vr}{\operatorname{Var}} %% variance
\newcommand{\rs}{X_1, X_2, \ldots, X_n} %%  random sample
\newcommand{\irs}{X_1, X_2, \ldots} %% infinite random sample
\newcommand{\rsd}{x_1, x_2, \ldots, x_n} %%  random sample, realised
\newcommand{\bX}{\boldsymbol{X}} %%  random sample, contracted form (bold)
\newcommand{\bx}{\boldsymbol{x}} %%  random sample, realised, contracted form (bold)
\newcommand{\bT}{\boldsymbol{T}} %%  Statistic, vector form (bold)
\newcommand{\bt}{\boldsymbol{t}} %%  Statistic, realised, vector form (bold)
\newcommand{\emv}{\hat{\theta}}
\DeclarePairedDelimiter\ceil{\lceil}{\rceil}
\DeclarePairedDelimiter\floor{\lfloor}{\rfloor}
\newcommand{\rpl}{\mathbb{R}_+}


% Title Page
\title{Exercícios: verossimilhança e suficiência}
\author{Disciplina: Inferência Estatística (MSc) \\ Instrutor: Luiz Carvalho}
\date{Junho/2022}

\begin{document}
\maketitle

\paragraph{Motivação:}
A suficiência é um dos pilares dos chamados ``princípios de redução de dados''; uma estatística suficiente encapsula toda a informação sobre um parâmetro e, portanto, deve ser preferida na busca por bons estimadores (veremos isso em detalhes em aulas posteriores).
Nesta lista de exercícios vamos trabalhar o conceito de suficiência e a sua conexão com a função de verossimilhança -- através do Teorema de Fatorização de Neyman-Fisher\footnote{Ronald Aylmer Fisher (1890--1962), biólogo e estatístico inglês. Jerzy Neyman (1894--1981), matemático e estatístico polonês.}.

\paragraph{Notação:} Como convenção adotamos $\mathbb{R} = (-\infty, \infty)$, $\rpl = (0, \infty)$ e $\mathbb{N} = \{1, 2, \ldots \}$.

\paragraph{Dos livros-texto:}

\begin{itemize}
    \item[a)] KN, Ch3: 4, 9a, 16a, 16b;
    \item[b)] CB, Ch6: 6.1, 6.7.
\end{itemize}


\paragraph{Extra:}

\begin{enumerate}
    \item Mostre que se $\rs \sim P_\theta$ são amostra aleatória de uma família paramétrica qualquer, então
    $$T(\rs) = (X_{(1)}, X_{(2)}, \ldots, X_{(n)}),$$
    é suficiente para $\theta$.
    \item Seja $\rs$ uma amostra aleatória de $P_\theta$ e $\rsd$ uma amostra observada (realizada).
    Para cada um dos modelos abaixo, apresente a função de verossimilhança uma estatística suficiente para $\theta$:
    \begin{enumerate}
        \item (\textbf{Weibull})
        \begin{equation*}
            f_\theta(x) = \theta_1\theta_2 x^{\theta_2-1}\exp\left(-\theta_1 x^{-\theta_2}\right)\mathbb{I}(x \in \rpl), (\theta_1, \theta_2) \in \rpl \times \rpl;
        \end{equation*}
        \item (\textbf{Pareto}) 
        \begin{equation*}
            f_\theta(x) = \frac{\theta_1\theta_2^{\theta_1}}{x^{\theta_1 + 1}}\mathbb{I}\left(x \in (\theta_2, \infty)\right), (\theta_1, \theta_2) \in \rpl \times \rpl;
        \end{equation*}
        \item (\textbf{Uniforme Discreta})
        \begin{equation*}
            P_\theta(X = x) = \frac{1}{\theta}\mathbb{I}(x \in \{1, 2, \ldots,\theta\}), \theta \in \mathbb{N};
        \end{equation*}
        \item (\textbf{Binomial negativa})
        \begin{equation*}
            P_\theta(X = x) = \binom{x + \theta_1 - 1}{\theta_1-1}(1-\theta_2)^x \theta_2^x, (\theta_1, \theta_2) \in \mathbb{N} \times (0, 1).
        \end{equation*}
    \end{enumerate}
    \item Dizemos que uma amostra aleatória tem distribuição gaussiana inversa se a densidade comum vale
    \begin{equation*}
        f_\theta(x) = \sqrt{\frac{\theta_1}{2\pi x^3}} \exp\left(\frac{\theta_1(x-\theta_2)^2}{2\theta_2^2x}\right)\mathbb{I}(x \in \mathbb{R}+),
    \end{equation*}
    para $(\theta_1, \theta_2) \in \rpl \times \mathbb{R}_{+}$.
    Encontre uma estatística suficiente nos seguintes casos:
    \begin{enumerate}
        \item $g(\theta) = \theta_1$ e $\theta_2$ é conhecido;
        \item $g(\theta) = \theta_2$ e $\theta_1$ é conhecido;
        \item $g(\theta) = (\theta_1, \theta_2)$, isto é, ambos são desconhecidos.
    \end{enumerate}
    \item \textbf{Desafio}: Seja $\rs$ uma amostra aleatória de uma distribuição $\operatorname{Uniforme}(0, \theta)$. 
    Mostre que $T(\rs) = \max(\rs)$ é suficiente para $\theta$ \underline{\textbf{sem}} usar o Teorema da Fatorização de Neyman-Fisher.
    
    \textbf{Dica:} Seja $\rs \sim P_\theta$ uma a.a. de uma família paramétrica qualquer.
    Se observamos a $i$-ésima estatística de ordem $X_{(i)} = a$, então:
    \begin{itemize}
        \item[i)] Para todo $j< i$, $X_{(j)} | X_{(i)}$ tem distribuição com densidade $f_\theta(x)/F_\theta(x)$;
        \item[ii)] Similarmente, para $k > i$, $X_{(k)} | X_{(i)}$ tem distribuição com densidade $f_\theta(x)/[1-F_\theta(x)]$.
    \end{itemize}
\end{enumerate}
% \newpage



% \bibliographystyle{apalike}
% \bibliography{refs}

\end{document}          
