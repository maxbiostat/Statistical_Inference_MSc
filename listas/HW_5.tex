\documentclass[a4paper,10pt, notitlepage]{report}
\usepackage[utf8]{inputenc}
\usepackage{natbib}
\usepackage{amssymb}
\usepackage{amsmath}
\usepackage{enumitem}
\usepackage{xcolor}
\usepackage{cancel}
\usepackage{mathtools}
\usepackage[portuguese]{babel}

%%%%%%%%%%%%%%%%%%%% Notation stuff
\newcommand{\pr}{\operatorname{Pr}} %% probability
\newcommand{\vr}{\operatorname{Var}} %% variance
\newcommand{\rs}{X_1, X_2, \ldots, X_n} %%  random sample
\newcommand{\ods}{X_{(1)}, X_{(2)}, \ldots, X_{(n)} } %%  ordered sample
\newcommand{\irs}{X_1, X_2, \ldots} %% infinite random sample
\newcommand{\rsd}{x_1, x_2, \ldots, x_n} %%  random sample, realised
\newcommand{\bX}{\boldsymbol{X}} %%  random sample, contracted form (bold)
\newcommand{\bx}{\boldsymbol{x}} %%  random sample, realised, contracted form (bold)
\newcommand{\bT}{\boldsymbol{T}} %%  Statistic, vector form (bold)
\newcommand{\bt}{\boldsymbol{t}} %%  Statistic, realised, vector form (bold)
\newcommand{\emv}{\hat{\theta}}
\DeclarePairedDelimiter\ceil{\lceil}{\rceil}
\DeclarePairedDelimiter\floor{\lfloor}{\rfloor}
\newcommand{\rpl}{\mathbb{R}_+}

% Title Page
\title{Exercícios: Estimação intervalar e quantidades pivotais}
\author{Disciplina: Inferência Estatística (MSc) \\ Instrutor: Luiz Carvalho}
\date{Agosto/2022}

\begin{document}
\maketitle

\paragraph{Motivação:} Estimativas pontuais não são adequadas a todas as situações.
Em várias aplicações queremos produzir estimativas que carreguem uma noção inerente de incerteza. 
Os intervalos de confiança são a ferramenta adequada para quantificar a incerteza sobre nossas inferências, ao mesmo tempo que vêm com garantias probabilísticas que fazem sentido dentro do paradigma clássico (frequentista).
As quantidades pivotais ou pivôs são transformações (dependentes dos parâmetros) cuja distribuição não depende dos parâmetros de interesse, e são extremamente úteis na construção de intervalos de confiança, sejam eles exatos ou aproximados.
Neste conjunto de exercícios vamos explorar as propriedades de pivôs e intervalos de confiança, tanto nas situações onde as quantidades são exatas quanto em situações de grandes amostras onde as propriedades se verificam aproximadamente.

\paragraph{Notação:} Como convenção adotamos $\mathbb{R} = (-\infty, \infty)$, $\rpl = (0, \infty)$ e $\mathbb{N} = \{1, 2, \ldots \}$.

\paragraph{Dos livros-texto:}

\begin{itemize}
    \item[a)] KN, Ch9:  9.1, 9.2, 9.3, 9.8, 9.13, 9.21, 9.33(a), 9.34.
    \item[b)] CB, Ch9: 11 (desafio), 14.
\end{itemize}

\paragraph{Extra:}

\begin{enumerate} 
\item Tome $\rs$ uma amostra aleatória de uma distribuição normal com média $\mu \in \mathbb{R}$ e variância $\sigma^2 \in \mathbb{R}_+$, ambas desconhecidas.
Defina $\bar{X}_n = \sum_{i=1}^n X_i/n$ e $S_n^2 = \sum_{i=1}^n (X_i -\bar{X}_n )/(n-1)$.

Derive a distribuição (incluindo a densidade com respeito a Lebesgue) das seguintes quantidades:
\begin{enumerate}
    \item Média padronizada:
    \begin{equation*}
        Z_n  := \sqrt{n}\frac{\bar{X}_n - \mu}{\sigma};
    \end{equation*}
    \item Variância escalonada:
    \begin{equation*}
        V_n := \frac{(n-1)S_n^2}{\sigma^2};
    \end{equation*}
    \item Quantidade $t$ de Student:
    \begin{equation*}
        T_n := \sqrt{n-1}\frac{Z_n}{\sqrt{V_n}}.
    \end{equation*}
\end{enumerate}
 \item (\textbf{Beta}) 
 Considere uma amostra aleatória $\rs$  de uma distribuição Beta com parâmetros $\theta$ (desconhecido) e $\beta = 1$, isto é com pdf comum
 \begin{equation*}
     f_\theta(x) = \theta x^{\theta-1}\mathbb{I}(x \in (0, 1)).
 \end{equation*}
 \begin{enumerate}
     \item Encontre a distribuição assintótica para o estimador de máxima verossimilhança de
      \begin{itemize}
          \item $g_1(\theta) = \theta$;
          \item $g_2(\theta) = \frac{\theta}{1+\theta}$.
      \end{itemize}
     \item Encontre uma quantidade pivotal  para $\theta$ e use-a para construir um intervalo de confiança de $(1-\alpha)\%$ para $\theta$.
 \end{enumerate}
  \item (\textbf{Beta escalonada})
  Considere uma amostra aleatória $\rs$  de uma distribuição com pdf comum
   \begin{equation*}
     f_\theta(x) = \frac{\theta x^{\theta-1}}{\tau^\theta}\mathbb{I}(x \in (0, \tau)),
 \end{equation*}
 com $\tau >0$ conhecida e $\theta > 0$ desconhecida.
 \begin{enumerate}
     \item Encontre o estimador $\delta_{\text{EMV}}(\bX_n)$ de máxima verossimilhança para $\theta$;
     \item Mostre que 
     \begin{equation*}
         T(\bX_n) := \frac{2n\theta}{\delta_{\text{EMV}}(\bX_n)}
     \end{equation*}
     é uma quantidade pivotal e encontre sua distribuição.
     Depois use $ T(\bX_n)$ para construir um intervalo de confiança exato de 
     \item Construa um intervalo de confiança assintótico (aproximado) de $(1-\alpha)\%$ para $\theta$.
     \item \textbf{[Computacional]} Compare os dois intervalos obtidos quanto à cobertura (o aproximado bate perto da cobertura nominal?) e largura dos intervalos obtidos.
 \end{enumerate}
 % \item \textbf{Desafio}:      
 % \textbf{Dica:} 
\end{enumerate}
% \newpage
% \bibliographystyle{apalike}
% \bibliography{refs}

\end{document}          

