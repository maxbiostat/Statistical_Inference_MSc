\documentclass[a4paper,10pt, notitlepage]{report}
\usepackage[utf8]{inputenc}
\usepackage{natbib}
\usepackage{amssymb}
\usepackage{amsmath}
\usepackage{enumitem}
\usepackage{xcolor}
\usepackage{cancel}
\usepackage{mathtools}
\usepackage[portuguese]{babel}

%%%%%%%%%%%%%%%%%%%% Notation stuff
\newcommand{\pr}{\operatorname{Pr}} %% probability
\newcommand{\vr}{\operatorname{Var}} %% variance
\newcommand{\rs}{X_1, X_2, \ldots, X_n} %%  random sample
\newcommand{\ods}{X_{(1)}, X_{(2)}, \ldots, X_{(n)} } %%  ordered sample
\newcommand{\irs}{X_1, X_2, \ldots} %% infinite random sample
\newcommand{\rsd}{x_1, x_2, \ldots, x_n} %%  random sample, realised
\newcommand{\bX}{\boldsymbol{X}} %%  random sample, contracted form (bold)
\newcommand{\bx}{\boldsymbol{x}} %%  random sample, realised, contracted form (bold)
\newcommand{\bT}{\boldsymbol{T}} %%  Statistic, vector form (bold)
\newcommand{\bt}{\boldsymbol{t}} %%  Statistic, realised, vector form (bold)
\newcommand{\emv}{\hat{\theta}}
\DeclarePairedDelimiter\ceil{\lceil}{\rceil}
\DeclarePairedDelimiter\floor{\lfloor}{\rfloor}
\newcommand{\rpl}{\mathbb{R}_+}


% Title Page
\title{Exercícios: Métodos de Estimação (Momentos, Máxima Verossimilhança)}
\author{Disciplina: Inferência Estatística (MSc) \\ Instrutor: Luiz Carvalho}
\date{Julho/2022}

\begin{document}
\maketitle

\paragraph{Motivação:} Em aula vimos os métodos de momentos, de mínimos quadrados e de máxima verossimilhança para obter estimativas (e estimadores!) de parâmetros e quantidades de interesse. Nesta lista você vai praticar encontrar esses estimadores e entender suas propriedades.

\paragraph{Notação:} Como convenção adotamos $\mathbb{R} = (-\infty, \infty)$, $\rpl = (0, \infty)$ e $\mathbb{N} = \{1, 2, \ldots \}$.

\paragraph{Dos livros-texto:}

\begin{itemize}
    \item[a)] KN, Ch9: 17a, 21a, 21b.
    \item[b)] CB, Ch7: 7.1, 7.2\footnote{Se precisar, peça ajuda ao monitor para a parte computacional.}, 7.6, 7.11, 7.12, 7.19, 7.37, 7.38.
\end{itemize}


\paragraph{Extra:}

\begin{enumerate}
    \item \textbf{Desafio}:  Seja $\rs$ uma amostra aleatória de uma família dominada, cuja densidade comum (com respeito a Lebesgue) é  $$ f_\theta(x) = \frac{\exp(-|x-\theta|)}{2}\mathbb{I}(x \in \mathbb{R}), \: \theta \in \mathbb{R}.$$
    Suponha que $n$ é par e encontre o estimator de máxima verossimilhança para $g(\theta) = \sin(\theta)/\theta$.
    Que peculiaridades tem esse estimador?
    
    \textbf{Dica:} Considere escrever a função de verossimilhança em termos da função sinal:
    \begin{equation*}
\operatorname{sgn}(x)=
\begin{cases}
-1,\quad x < 0,\\
\:\:\:0, \quad x = 0,\\
\:\:\:1, \quad x > 0.
\end{cases}
    \end{equation*}
    para $x \in \mathbb{R}$.
\end{enumerate}
% \newpage



% \bibliographystyle{apalike}
% \bibliography{refs}

\end{document}          
